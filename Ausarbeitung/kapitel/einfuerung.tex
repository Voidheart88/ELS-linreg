\chapter{Einleitung} %Aufgabenstellung, Problematik und Zielsetzung der Arbeit als Vorausschau (1-2 Seiten)
    Diese Arbeit hat das Ziel regenerative Energieformen für mehrere Standorte auf ihre Anwendbarkeit zu überprüfen.
    Die Bedeutung der dezentralen Energieerzeugung aus regenerativen Energiequellen steigt immer weiter an und die Energieformen sind stark standort- und witterungsabhängig.
    Zusätzlich zur Abhängigkeit von klimatischen Gegebenheiten gilt es vorhandene Flächen bestmöglich auszunutzen.
    Der Ausbau regenerativer Energieerzeugungsformen muss sich daher sowohl an ökologischen als auch an technischen und ökonomischen Kriterien messen lassen.

    Diese Arbeit soll den Aspekt der technischen Machbarkeit näher beleuchten.
    Es soll ein Vergleich der Machbarkeit verschiedener Energieerzeugungsformen gegeben werden.
    Hierzu muss ein Modell einer Wohneinheit entwickelt werden, der Energieverbrauch bestimmt und auf Basis dieser Daten die Auslegung der Generatoren vorgenommen werden.
    Im zweiten Kapitel wird ein Überblick über den Stand der Technik verschiedener Erzeugungs und Speicherformen gegeben und im dritten Kapitel die Erstellung des Modells sowie die Auslegung der Anlagenbestandteile vorgenommen.
    Das vierte Kapitel wird ein Fazit aus den Ergebnissen der Daten ziehen und einen Überblick über Möglichkeiten der Verfeinerung des Modells geben.
    

    