\chapter{Auslegung der Anlagenteile} 

Um den Standort mit Energie zu versorgen ist es notwendig, die erzeugte Energie mittels Wechselrichter in die Netzspannung von 230\,V zu wandeln.
Abb.~\ref{fig:Blockschaltbild} zeigt ein Blockschaltbild der verwendeten Architektur.
Der Wechselrichter ist durch seinen Aufbau in der Lage, die erzeugte Energie wahlweise in das Netz, oder in den Energiespeicher zu speisen.
In energieärmeren Perioden ist es möglich, die im Energiespeicher enthaltene Energie wieder in das Inselnetz einzuspeisen.
\begin{figure}[h]
  \centering
  \includegraphics[width=7cm]{src/Schaltungen/EVersBlock.pdf}
  \caption{Blockschaltbild Energieversorgung}\label{fig:Blockschaltbild}
\end{figure}

\section{Modellierung des Gebäudes}
Ziel dieses Abschnittes ist es ein autarkes Versorgungssystem für ein Wohngebäude, für 2 Personen, zu modellieren.

Der durchschnittliche Gesamtenergiebedarf eines Haushaltes in Deutschland betrug im Jahr 2017, 16.037\,kWh\cite{Webp:DestatisEnergieverbrauch}.
Davon wurden 69\,\% für Raumwärme, 15\,\% für die Warmwasserbereitung, 5,8\,\% für das Kochen, 
Trocknen und Bügeln, 8,6\,\% für Haushaltsgeräte wie Fernseher und Computer und 1,6\,\% für die Beleuchtung verwendet\cite{Webp:DestatisEnergieverbrauch}.
Der durchschnittliche Wärmebedarf, errechnet sich aus dem Aufwand für die Raumwärmeerzeugung und Warmwasserbereitung. Er beträgt 13.471,08\,kWh je Haushalt.
Der elektrische Energiebedarf beträgt somit als Rest 2.565,92\,kWh je Haushalt.
Für die Bestimmung des jährlichen Gesamtenergiebedarfes wird angenommen das sich der elektrische Energiebedarf aus dem Lebensstandart der Bewohner ergibt und deshalb ortsunabhängig ist.
Der Heizbedarf hingegen ist abhängig vom Standort, weshalb an dieser Stelle ein Berechnungsmodell notwendig ist (Siehe Abb~\ref{fig:THERMM}).
\begin{figure}[h]
  \centering 
  \includegraphics[width=6cm]{bilder/ThermModell.pdf}
  \caption[Thermisches Modell]{Thermisches Modell}\label{fig:THERMM}
\end{figure}

Eine durchschnittliche Wohneinheit in Deutschland hat eine Fläche von \(\mathrm{94,1\,m^2}\) und 2,1 Bewohner.\cite{Webp:DestatisWohnungsflaeche}
Die meisten Bauordnungen fordern für ein Vollgeschoss eine Höhe von 2,3\,m\cite{Webp:stmb-vollgeschoss}.
Für das Modell wird, an dieser Stelle, die Wohnfläche auf 2 Stockwerke aufgeteilt.
Die Maße der Seitenwände des Gebäudemodells ergeben sich aus der Grundfläche eines Stockwerkes:
\[s_\mathrm{l} = s_\mathrm{b} = \sqrt{A_\mathrm{etage}} = \sqrt{\mathrm{\frac{94,1\,m^2}{2}}} \mathrm{\approx 6,9\,m}\]
Der Wärmedurchgang zwischen den Geschossen wird als idealleitend betrachtet und 
die Außenwände, zum Zwecke der Vereinfachung der Berechnung, als 36.5\,cm, angenommen.
Es entsteht damit ein Quader mit den Maßen: \(s_\mathrm{l} = \mathrm{6,9\,m},s_\mathrm{b} = \mathrm{6,9\,m},s_\mathrm{h} = \mathrm{4,6\,m}\).
Aus der Fläche des Quaders (Wände, Decke, Fundament) lässt sich der Wärmewiderstand des Innenraums zur Umgebung bestimmen.
\[A_\mathrm{Geb} = 4\cdot A_\mathrm{Wand} + 2\cdot A_\mathrm{De\,Fu} = 4\cdot s_\mathrm{l} \cdot s_\mathrm{h} + 2\cdot s_\mathrm{l} \cdot s_\mathrm{b} \mathrm{= 4 \cdot 6,9\,m \cdot 4,6\,m + 2 \cdot 6,9\,m \cdot6,9\,m = 222,18\,m^2}\]
Das Maß für die Verlustleistung gibt der sogenannte ``U-Wert'' oder ``Wärmedurchgangskoeffizient'' \(\lambda\) (Vgl. DIN EN ISO 6946).
Dieser beträgt für eine in Deutschland übliche Ziegelwand mit 36.5\,cm Wandstärke \(\mathrm{0,8\,\frac{W}{K\cdot m^2}}\)
Damit lässt sich der Wärmewiderstand des Gebäudemodells durch folgenden Zusammenhang beschreiben:
\[R_\mathrm{\vartheta} = A * \lambda = 222,18\,m^2 \cdot \mathrm{0,8\,\frac{W}{K\cdot m^2}} = 177,744\,\frac{W}{K}\]

\subsubsection{Energiebedarf}
  Die notwendige Heizleistung für einen gegebenen Körper, ergibt sich aus folgender Gleichung:
  \[\dot{Q} = R_\mathrm{\vartheta} \cdot (T_\mathrm{2} - T_\mathrm{1})\]
  Wobei \(T_\mathrm{2}\) die Temperatur des Körpers, \( T_\mathrm{1})\) die Temperatur der Umgebung und \(R_\mathrm{\vartheta}\) den Wärmewiderstand zwischen beiden Potentialen beschreibt.
  Da in Zeiten in denen die Außentemperatur größer als \(21\,^\circ C\) ist, nicht geheizt werden muss, muss die Gleichung noch modifiziert werden:
  \[
  \dot{Q} = \left\{\begin{array}
    {rcl} T_1 > 21\,^\circ C&:& 0\\
    T_1 \leq 21\,^\circ C&:& R_\mathrm{\vartheta} \cdot (21\,^\circ C- T_1)
  \end{array}\right.
  \]
  Der Gesamtenergiebedarf pro Jahr kann nun durch folgenden Zusammenhang ermittelt werden:
  \[E_\mathrm{ges} = \sum_{i = 1}^{12}(\dot{Q_i} \cdot \frac{731\,h}{a \cdot 1000\frac{W}{kW}})\]
  Zur Verifikation wurde der Standort Mannheim herangezogen (Siehe Abb~\ref{fig:WTRMAN}).
  Als Gesamtenergiebedarf wurde ein Wert von \(\mathrm{16.680\,\frac{kWh}{a}}\) ermittelt.
  Unter dem Aspekt, dass in dieser Berechnung auch die kühleren Jahre anfang des Jahrzehnts berücksichtigt wurden und dem Aspekt, dass sich die obiger Durchschnitt für den Heizbedarf auf das Jahr 2017 bezieht, erscheint dieser Wert plausibel.
  Als Gesamtheizenergiebedarf für den Standort Oulu wurden \(\mathrm{27.531\,\frac{kWh}{a}}\) und als Heizenergiebedarf für den Standort Alice Springs wurden \(3.526\,\frac{kWh}{a}\) ermittelt.
  Der Gesamtenergiebedarf beträgt deshalb für den Standort Oulu \(\mathrm{30096,92\,\frac{kWh}{a}}\) und für den Standort Alice Springs \(\mathrm{6091,92\,\frac{kWh}{a}}\).
  \begin{figure}[h]
    \centering 
    \includegraphics[width=7cm]{src/WetterMannheim/MannheimTemp.pdf}
    \includegraphics[width=7cm]{src/WetterMannheim/MannheimHeiz.pdf}
    \caption[Temperaturen Mannheim]{Temperatur und Heizenergie Mannheim, Quelle:\cite{Webp:WWOMAN}}\label{fig:WTRMAN}
  \end{figure}

  Für die Auslegung der Speicher wird zusätzlich noch der monatliche Bedarf ermittelt werden.
  Hierfür wird der elektrische Energiebedarf auf die zwölf Monate verteilt und auf den Heizenergiebedarf addiert.
  \[E_\mathrm{bed} = E_\mathrm{mon} + \frac{1}{12} E_\mathrm{el}\]
  Abb.~\ref{fig:EBED} zeigt die ermittelten Bedarfe auf die Monate verteilt.
  \begin{figure}
    \centering
    \includegraphics[width=7cm]{src/WetterAliceSprings/alice_need.pdf}
    \includegraphics[width=7cm]{src/WetterOulu/oulu_need.pdf}
    \caption{Energiebedarf Oulu und Alice Springs}\label{fig:EBED}
  \end{figure}

\section{Auslegung einer Solarenergieversorgung}
  Die erzeugte Energie des Solargenerators muss gleich oder höher der benötigten Energie für die Wärme und Stromversorgung sein.
  Die dafür notwendige installierte Leistung ergibt sich deshalb aus dem Energieangebot je Fläche am gewählten Standort.
  Hierfür wird eine Berechnung anhand der Sonnenstunden gewählt.
  Als Sonnenstunde wird eine Zeitspanne bezeichnet, während der die flächenbezogene Leistungsdichte senkrecht zur Sonnenrichtung mindestens \(E_e = \mathrm{120\,\frac{W}{m^2}}\) beträgt.
  Diese Berechnungsmethode gibt einen Minimalwert, für die Strahlungsleistung der an einem Ort zu erwarten ist, an.
  Die mittlere Summe der Sonnenstunden beträgt für den Ort Oulu \(T_\mathrm{Oulu} = \mathrm{2146.2\,h}\) und für Alice Springs, \(T_\mathrm{AlSp} = \mathrm{3555.6\,h}\).
  Die Minimalenergie je Quadratmeter ergibt sich folglich zu:
  \[\rho_\mathrm{\min} = E_e \cdot T\]
  Der Energieertrag aus der Energiedichte muss zusätzlich aufgrund des Wirkungsgrads reduziert werden:
  \[\rho_\mathrm{sol} = \rho_\mathrm{\min} \cdot \frac{\eta_\mathrm{sol}}{100\,\%}= \rho_\mathrm{\min} \cdot \frac{19,5\,\%}{100\,\%}\] 
  Die notwendige Fläche ergibt sich aus dem Energiebedarf des Objektes und der flächenbezogenen Energiedichte:\\
  \(E_{Oulu} = \mathrm{30096,92\,\frac{kWh}{a}}\), \(E_{AlSp} = \mathrm{6091,92\,\frac{kWh}{a}}\)\\
  \[A = \frac{E_\mathrm{bedarf}}{\rho_\mathrm{sol}}\]
  \[A_\mathrm{Oulu} = \mathrm{\frac{30096,92\,\frac{kWh}{a}}{\mathrm{120\,\frac{W}{m^2}} \cdot 0.195 \cdot 2146.2\,\frac{h}{a}} = 599.288\,m^2}\]
  \[A_\mathrm{AlSp} = \mathrm{\frac{6091,92\,\frac{kWh}{a}}{\mathrm{120\,\frac{W}{m^2}} \cdot 0.195 \cdot 3555.6\,\frac{h}{a}} = 73\,m^2}\]

  \subsection{Sensitivitätsanalyse - Solar}
  Die Sensitivitätsanalyse soll zeigen, welchen Einfluss die Veränderung der Parameter auf das Ergebnis hat.
  Für die Solarenergieversorgung sind die Parameter der Funktion, einerseits die Anzahl der Sonnenstunden und andererseits der Wirkungsgrad der verwendeten Zellen.
  Die Wirkungsgrade von Solarzellen betragen je nach Technologie zwischen 15\,\% (CIS-Zelle) und 30\,\% (Konzentrator-Zelle)\cite{BK:QEEK}
  Die Maximalzahl von Sonnenstunden erreicht der Ort Yuma im Staat Arizona in Nordamerika\cite{Webp:YAY}.
  Für eine feste Anzahl an Sonnenstunden von 2500\,h im Jahr und einen variablen Wirkungsgrad \(\eta_\mathrm{sol}\), ergibt sich, im Intervall von \( \eta_\mathrm{sol} \in \mathrm{[0.15\,\%;30\,\%]}\), 
  eine Funktion für den Flächenbedarf pro \(\mathrm{kWh/a}\) Energiebedarf die umgekehrt proportional zum Wirkungsgrad ist.
  Für einen festen Wirkungsgrad \( \eta_\mathrm{sol} = 19\,\% \) und eine variable Zahl Sonnenstunden, ergibt sich, im Intervall von \( T_\mathrm{sol} \in \mathrm{[1500\,h;4000\,h]}\), 
  eine Funktion für den Flächenbedarf pro \(\mathrm{kWh/a}\) Energiebedarf die direkt proportional zur Anzahl der Sonnenstunden ist.

\section{Auslegung einer Windenergieversorgung}
  Die erzeugte Energie eines Windgenerators muss gleich oder höher der benötigten Energie für die Wärme und Stromversorgung sein.
  Die dafür notwendige installierte Leistung ergibt sich aus dem Windangebot am gewählten Standort und der Kennlinie des Windgenerators.
  Abb.~\ref{fig:WINDKENN} zeigt beispielhaft eine Kennlinie für einen 16\,kW Windgenerator der Firma Britwind.
  \begin{figure}
    \centering
    \includegraphics[width=7cm]{src/pwr/PowerPerWind-crop.pdf}
    \caption{Kennlinie 16kW Export Quelle:\cite{Datasheet:BRITWIND}}\label{fig:WINDKENN}
  \end{figure}
  Mit den, aus Wetterdaten bekannten, durchschnittlichen monatlichen Windgeschwindigkeiten der letzten zehn Jahre, 
  wird die zu erwartende Leistung und Energie je Monat berechnet. 
  Die Summe dieser Werte ergibt die zu erwartende gewonnene Energie je Windturbine über ein Jahr. 
  In Alice Springs wird, bei Nutzung einer H15 Class II 16kW, die gewonnene Energie je Monat in voraussichtlich 2 Monaten
  nicht ausreichen, in diesen Monaten muss zusätzlich Energie aus dem Speicher genutzt werden.
  Über das ganze Jahr hinweg, kann jedoch der Energiebedarf des Haushalts fast dreifach gedeckt werden. 
  In Oulu müssen, bei Nutzung der H15 Class IV 12kW Windturbine, 
  4 Windturbinen aufgestellt werden, um den jährlichen Energieverbrauch zu decken.
   
\section{Auslegung Energiespeicher}
  Die Auslegung eines Inselsystems muss sich an dem Monat mit dem geringsten Energieangebot orientieren. Dies betrifft sowohl die Generatorfläche als auch die Ausrichtung der Module, sowie die Größe der Windgeneratoren. 
  Dies führt zu einer deutlichen Überdimensionierung des Systems, in den einstrahlungs- beziehungsweise windstärkeren Monaten.
  \newpage
  \subsection{Energiespeicher für Solargenerator}
  \begin{figure}
    \centering
    \includegraphics[width=7cm]{src/WetterAliceSprings/alice_bilanz_solar.pdf}
    \includegraphics[width=7cm]{src/WetterOulu/Oulu_bilanz_solar.pdf}
    \includegraphics[width=7cm]{src/WetterAliceSprings/alice_bilanz_wind.pdf}
    \includegraphics[width=7cm]{src/WetterOulu/Oulu_bilanz_wind.pdf}
    \caption{Energiebilanzen Oulu,Alice Springs Solar/Wind}\label{fig:SpeicherSol}
  \end{figure}
  Der Auslegung für einen Solarspeicher wird aufgrund der Energiebilanz vorgenommen.
  Der Speicher muss mindestens eine Größe besitzen die notwendig ist, um das die längste Defizitperiode \(n\) als Energie Speichern zu können.
  \[E_\mathrm{Speicher} = \sum_{i=1}^{n} E_{defizit}\]\\

  Für den Standort Oulu wurde ein Speicher mit mindestens der Größe von: \(\mathrm{15808.06\,kWh}\) bestimmt.\\
  Für den Standort Alice Springs wurde ein Speicher mit mindestens der Größe von: \(\mathrm{2471.92\,kWh}\) bestimmt.
  
  \subsection{Energiespeicher für Windgenerator}
  Der Auslegung für einen Windenergiespeicher wird wieder aufgrund der Energiebilanz nach obiger Gleichung vorgenommen.
  Für den Standort Oulu wurde ein Speicher mit mindestens der Größe von: \(\mathrm{1812\,kWh}\) bestimmt.
  Für den Standort Alice Springs wurde ein Speicher mit mindestens der Größe von: \(\mathrm{665\,kWh}\) bestimmt.
  