% Author: Dominik Stolte
% Beamer presentation for renewable energys 2
% 3.1.2020

\documentclass{beamer}
\usepackage[utf8]{inputenc}
\usepackage[font=scriptsize]{caption}
\usepackage{bookmark}
\usepackage{verbatim}
\usepackage{textpos}

\title{MERLIN -- Methane Remote Lidar Mission}
\author{Dominik Stolte}
\institute[{Hochschule Mannheim}]{Hochschule Mannheim}
\logo{\includegraphics[height=0.8cm]{Bilder/hsma-logo.pdf}\vspace{235pt}}
\date{\today}

\begin{document} 
\setbeamertemplate{itemize items}[circle]

\begin{frame}
  \titlepage{}
\end{frame}

\begin{frame}
  \frametitle{Warum das Ganze?}
  \begin{figure}[t]
    \centering
    \includegraphics[width=4cm]{Bilder/laser-meme.jpg}
    \includegraphics[width=3.5cm]{Bilder/Castle_Bravo.jpg}
    \caption{Warum das ganze?\\ 
    Quellen: Memereaktor/US Gov}\label{fig:WHY}
  \end{figure}
\end{frame}

\begin{frame}
  \frametitle{Inhalt}
  \tableofcontents
\end{frame}

\section{Einführung}
\begin{frame}
  \frametitle{MERLIN}    
  \begin{figure}[t]
    \centering
    \includegraphics[width=4cm]{Bilder/Forschung_MERLIN_fig1.png}
    \caption{MERLIN - künst. Darstellung\\ 
    Quelle: DLR - Institut f. Atmosphärenphysik}\label{fig:MERLIN}
  \end{figure}
\end{frame}

\subsection{Methan}
\begin{frame}
  \frametitle{Methan}
  \begin{itemize}
    \item Was ist Methan
    \item Methandetektion
    \item IPDA
  \end{itemize}
\end{frame}

\begin{frame}
  \frametitle{Methan}
  \begin{figure}[t]
    \centering
    \includegraphics[width=4cm]{Bilder/meth-formula.pdf}
    \caption{Methan - Struktur\\ 
    Quellen: Wikipedia}\label{fig:METH}
  \end{figure}
  \begin{itemize}
    \item Klimagas
    \item Entsteht bei Anäroben Stoffwechselprozessen - Fäulnis
    \item Entsteht auch durch vulkanische Aktivität
  \end{itemize}
\end{frame}

\begin{frame}
  \frametitle{Methan II - Konzentration}
  \begin{itemize}
    \item $\mathrm{CO_2}$-Anteil: 280 ppm vorindustriell - 400 ppm jetzt (+42\%)
    \item $\mathrm{CH_4}$-Anteil: 0,73 ppm vorindustriell - 1,8 ppm jetzt (+148\%)
    \item Klimaschädlichkeit 25x höher als bei $\mathrm{CO_2}$ (IPCC)
  \end{itemize}
\end{frame}

\begin{frame}
  \frametitle{Methan III - Vorkommen}
  \begin{itemize}
    \item Gebunden als Methanhydrat
    \item Hauptbestandteil von Erdgas
    \item Bestandteil von Grubengas in Kohlegruben
  \end{itemize}
\end{frame}

\subsection{Methandetektion}
\begin{frame}
  \frametitle{Wie findet man Methan?}
  \begin{itemize}
    \item Passiv - Sonne strahlt in Atmosphäre und man guckt hin
    \begin{itemize}
      \item günstig
      \item benötigt wenig Energie
      \item geringe Auflösung
      \item geht nur bei Tag
    \end{itemize}
    \item Aktiv - LASER :D
    \begin{itemize}
      \item teuer
      \item benötigt viel Energie
      \item hohe Auflösung
      \item geht immer
    \end{itemize}
  \end{itemize}
\end{frame}

\begin{frame}
  \frametitle{Integrated path differential absorption}
  \begin{figure}[t]
    \centering
    \includegraphics[width=4cm]{Bilder/Forschung_MERLIN_fig2.png}
    \caption{aktive Messung\\ 
    Quelle: DLR}\label{fig:IPDA}
  \end{figure}

  \begin{itemize}
    \item $\lambda_{\mathrm{on}}$ - 1645.552\,nm
    \item $\lambda_{\mathrm{off}}$ - 1645.846\,nm
  \end{itemize}  
\end{frame}

\begin{frame}
  \frametitle{Merlin - Vorgänger}
  \begin{itemize}
    \item High Altitude and Long Range Research Aircraft - HALO
    \item Environmental Satellite - ENVISAT
    \item Greenhouse Gases Observing Satellite - GOSAT
  \end{itemize}
\end{frame}

\section{Technik}
\begin{frame}
  \frametitle{Technik}
  \begin{itemize}
    \item Orbit
    \item Laser
    \item Energiegewinnung 
    \item Housekeeping
  \end{itemize}
\end{frame}

\subsection{Orbit}
\begin{frame}
  \frametitle{Merlin - Orbit}
  \begin{figure}[t]
    \centering
    \includegraphics[width=6cm]{Bilder/merlin_orbit.pdf}
    \caption{Orbit Merlin\\ 
    Quelle: eigenes Bild}\label{fig:MERLORBIT}
  \end{figure}
\end{frame}

\begin{frame}
  \frametitle{Merlin - Orbit II}
  \begin{itemize}
    \item $T_{\mathrm{UM}} = \frac{r_{\mathrm{Erde}}+h_{\mathrm{Orbit}}}{V_{\mathrm{Umlauf}}}\cdot \frac{2\pi}{60\frac{\mathrm{s}}{\mathrm{min}}} \approx 102,79\,min$
    \item Blickrichtung Sonne - Es wird nie Nacht für den Satelliten!
    \item Alle 28 Tage ist die Erde einmal kartiert
    \item Sichtbarkeit 15\,min pro Tag
    \item 500\,km über NN
    \item Local Time of Ascending Node (LTAN) 6:00
  \end{itemize}
\end{frame}

\subsection{Laser}

\begin{frame}
  \frametitle{Merlin - Laser I}
  \begin{figure}[t]
    \centering
    \includegraphics[width=6cm]{Bilder/Laser2.jpg}
    \caption{So machts Merlin\\ 
    Quelle: Fraunhofer ILT, Aachen}\label{fig:MERLASER}
  \end{figure}
\end{frame}

\begin{frame}
  \frametitle{Merlin - Laser II}
  \begin{figure}[t]
    \centering
    \includegraphics[width=6cm]{Bilder/MERLIN_tech.pdf}
    \caption{So machts Merlin\\ 
    Quelle: eigenes Bild}\label{fig:MERLASER}
  \end{figure}
\end{frame}

\begin{frame}
  \frametitle{Merlin - Laser }
  \begin{figure}[t]
    \centering
    \includegraphics[width=6cm]{Bilder/OPO.jpg}
    \caption{Optisch parametrischer Oszillator\\ 
    Quelle: Fraunhofer ILT, Aachen}\label{fig:MERLASER}
  \end{figure}
\end{frame}


\subsection{Energie}
\begin{frame}
  \frametitle{Merlin - Energie}
  \begin{itemize}
    \item Satellit gesamt - 400\,W
    \item davon Nutzlast - 150\,W
    \item Rest: Margin und Kommunikation, Housekeeping, Lagereglung
  \end{itemize}
\end{frame}

\begin{frame}
  \frametitle{Merlin - Energie II} 
  \begin{figure}[t]
    \centering
    \includegraphics[width=4cm]{Bilder/Forschung_MERLIN_fig1.png}
    \caption{MERLIN - künst. Darstellung\\ 
    Quelle: DLR - Institut f. Atmosphärenphysik}
  \end{figure}
\end{frame}

\subsection{Bus und Kommunikation}
\begin{frame}
  \frametitle{Merlin - Träger} 
  \begin{figure}[t]
    \centering
    \includegraphics[width=7cm]{Bilder/merlin_satbus.jpeg}
    \caption{MERLIN - Satellitenbus\\ 
    Quelle: CNES}
  \end{figure}
\end{frame}

\begin{frame}
  \frametitle{Merlin - Lagereglung} 
  \begin{figure}[t]
    \centering
    \includegraphics[width=4cm]{Bilder/reaktionsrad.pdf}
    \caption{Funktion Reaktionsrad\\ 
    Quelle: eigenes Bild}
  \end{figure}
  \[J = \int_V \vec{r}^2 p(\vec{r})\mathrm{d}V\]
  \[L = J \cdot \omega\]
  \[M = J \cdot \dot{\omega}\]
  \[M = k_{\mathrm{M}}\cdot I\]
\end{frame}

\begin{frame}
  \frametitle{Merlin - Lagereglung II }
  \begin{itemize}
    \item Oft magnetgelagert - keine Reibung
    \item Reaktionsrad sättigt durch Benutzung
    \item Entsättigung durch Zünden von Steuerdüsen
    \item Entsättigung durch magnetisches "Festklammern" 
  \end{itemize}
\end{frame}

\section{Zusammenfassung}
\begin{frame}
  \frametitle{Gelerntes}
  \begin{itemize}
    \item Was ist Methan ?
    \item Wo kommt Methan her ?
    \item Wie finded man Methan ?
    \item Was ist IPDA ?
    \item Warum nutzt man einen solarsynchronen Orbit ?
    \item Wie funktioniert die Energieversorgung ?
    \item Wie ist der Satellit grob aufgebaut ?
    \item Wie funktioniert Lagereglung bei einem Satelliten ?
  \end{itemize}
\end{frame}

\begin{frame}
  \frametitle{Ende}
  \begin{center}
    \large{Fragen?}
  \end{center}
\end{frame}

\begin{frame}
  \frametitle{Ende II}
  \begin{center}
    \large{Fragen?}
  \end{center} 
  \begin{figure}[t]
    \centering
    \includegraphics[width=8cm]{Bilder/star-wars_2785213.jpg}
    \caption{Der Todesstern\\ 
    Quelle:\\ https://www.gamestar.de/artikel/star-wars-neue-details-zum-todesstern-stellen-alles-auf-den-kopf,3308561.html}
  \end{figure}
\end{frame}

\begin{frame}
  \frametitle{Quellen}
  \begin{itemize}
    \item Wiki: \url{https://de.wikipedia.org/wiki/Merlin_(Satellit)}
    \item DLR: \url{https://www.dlr.de/rd/desktopdefault.aspx/tabid-2440/3586_read-31672/}
    \item Earth Observation: \url{https://directory.eoportal.org/web/eoportal/satellite-missions/m/merlin}
    \item Fraunhofer Institut: \url{https://www.ilt.fraunhofer.de/de/presse/pressemitteilungen/pm2017/pressemitteilung-28-04-2017.html}
  \end{itemize}
\end{frame}

\end{document}
